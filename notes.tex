\documentclass[10pt,a4paper]{article}
\usepackage[margin=1in, includeheadfoot]{geometry}

\usepackage[dvipsnames]{xcolor}
\usepackage{amsmath}
\usepackage{amssymb}
\usepackage{amsthm}
%\usepackage{makeidx}
\usepackage{thmtools}
\usepackage{graphicx}
\usepackage{hyperref}
\usepackage{tikz}
\usepackage{algpseudocode}
\usepackage{multirow}
\usepackage{framed}
\usepackage{cleveref} % allow for \cref command

\usepackage[most]{tcolorbox}


\usetikzlibrary{positioning}

\usepackage{thmtools}
%\usepackage[skip=6pt]{parskip}

\usepackage{fancyhdr}
\pagestyle{fancy}

\fancyhead[L]{\textbf{OPRF Lower Bound}}
\fancyhead[R]{\thepage}
\fancyfoot{}

\setlength\parindent{0pt}

\hypersetup{
	colorlinks=true,
	linkcolor=red,
	filecolor=red,  
	citecolor=red,    
	urlcolor=blue,
	pdftitle={Games on Graphs \#1},
	pdfpagemode=FullScreen,
}

% \usepackage[framemethod=tikz]{mdframed}

\declaretheorem[name=Theorem, numberwithin=section]{theorem}
\declaretheorem[name=Lemma, numberwithin=section]{lemma}
\declaretheorem[name=Inequality, numberwithin=section]{inequality}
\declaretheorem[name=Definition, numberwithin=section]{definition}

\usepackage[english]{babel}
\title{\textbf{OPRF Lower Bound}}
\author{Jake Januzelli, Naman Kumar, Mike Rosulek}

%------- General Macros

\newcommand{\np}{\mathbf{NP}}
\newcommand{\p}{\mathbf{P}}
\newcommand{\ip}{\mathbf{IP}}
\newcommand{\pspace}{\mathbf{PSPACE}}
\newcommand{\am}{\mathbf{AM}}
\newcommand{\ma}{\mathbf{MA}}
\newcommand{\F}{\mathcal{F}}
\newcommand{\field}{\mathbb{F}}
\newcommand{\E}{\mathbb{E}}
\newcommand{\rel}{\mathcal{R}}
\newcommand{\lang}{\mathcal{L}}
\newcommand{\query}{\mathcal{Q}}
\newcommand{\prover}{\mathcal{P}}
\newcommand{\verifier}{\mathcal{V}}
\newcommand{\decision}{\mathcal{D}}
\newcommand{\simulator}{\mathcal{S}}
\newcommand{\sat}{\mathsf{SAT}}
\newcommand{\nc}{\mathbf{NC}}
\newcommand{\hvzk}{\mathbf{HVZK}}
\newcommand{\zk}{\mathbf{ZK}}
\newcommand{\pzk}{\mathbf{PZK}}
\newcommand{\czk}{\mathbf{CZK}}
\newcommand{\szk}{\mathbf{SZK}}
\newcommand{\pc}{\mathsf{pc}}
\newcommand{\vc}{\mathsf{vc}}
\newcommand{\vr}{\mathsf{vr}}
\newcommand{\poly}{\mathsf{poly}}
\newcommand{\add}{\mathsf{add}}
\newcommand{\mul}{\mathsf{mul}}
\newcommand{\view}{\mathsf{View}}
\newcommand{\trace}{\textsc{Trace}}
\newcommand{\bpp}{\mathbf{BPP}}
\newcommand{\hilbert}{\mathcal{H}}
\newcommand{\ket}[1]{\left|#1\right\rangle}
\newcommand{\bra}[1]{\left\langle#1\right|}
\newcommand{\braket}[2]{\langle#1|#2\rangle}
\newcommand{\puncture}{\mathsf{Puncture}}

%---------- Macros

\newcommand{\env}{\mathcal{Z}}
\newcommand{\adv}{\mathcal{A}}
\newcommand{\red}{\mathcal{R}}
\newcommand{\pake}{\mathcal{F}}
\newcommand{\user}{\mathsf{User}}
\newcommand{\sk}{\mathsf{sk}}
\newcommand{\pw}{\mathsf{pw}}
\newcommand{\VK}{\mathsf{VK}}
\newcommand{\SK}{\mathsf{SK}}
\newcommand{\crs}{\mathsf{crs}}
\newcommand{\newsession}{\mathsf{NewSession}}
\newcommand{\testpwd}{\mathsf{TestPwd}}
\newcommand{\compromised}{\mathsf{compromised}}
\newcommand{\server}{\mathsf{Server}}
\newcommand{\msg}[1]{\mathsf{msg}_{#1}}
\newcommand{\rgets}{\xleftarrow{\$}}
\newcommand{\G}{\mathbb{G}}
\newcommand{\ct}{\mathsf{ct}}
\newcommand{\st}{\mathsf{st}}
\newcommand{\sign}{\mathsf{Sign}}

% --------------- OPRF
\newcommand{\oprf}{\mathsf{OPRF}}
\newcommand{\prf}{\mathsf{PRF}}
\newcommand{\bin}{\{0,1\}}
\newcommand{\secpar}{\lambda}
\newcommand{\sender}{\mathcal{S}}
\newcommand{\receiver}{\mathcal{R}}
\newcommand{\ot}{\mathsf{OT}}

\def\mike#1{\textcolor{blue}{Mike: #1}}
\def\naman#1{\textcolor{red}{Naman: #1}}
\def\jake#1{\textcolor{ochre}{Jake: #1}}

% --------------- BOXES
\newtcolorbox[auto counter]{funcbox}[3]{enhanced, breakable, title={#1}, colbacktitle=white, coltitle=black, attach boxed title to top left={yshift=-2mm,xshift=2mm}, label={#2}, nameref={#3},colback=white}

\begin{document}
	\maketitle

\section{Definitions}

Let $\prf:\bin^\secpar\times\bin^{m(\secpar)}\rightarrow\bin^{n(\secpar)}$ be a pseudorandom function with stretch $n\in\poly(\secpar)$. We define the OPRF Functionality $\F_\oprf$ as follows.

\begin{funcbox}{OPRF Functionality $\F_\oprf$}{}{}
	\textbf{Inputs.} $\sender$ has input OPRF key $k\in\bin^\secpar$, $\receiver$ has input some $x\in\bin^{m(\secpar)}$ in the domain of the PRF.\\
	\textbf{Outputs.} $\receiver$ gets $\prf_k(x)$.
\end{funcbox}

We further define the OT functionality as below.

\begin{funcbox}{OT Functionality $\F_\ot$}{}{}
	\textbf{Inputs.} $\sender$ has input two strings $(m_0, m_1)\in\bin^{\poly(\secpar)}$ while receiver has a bit $b$.\\
	\textbf{Outputs.} $\receiver$ gets $m_b$.
\end{funcbox}

\section{Proof of Insecurity of `Trivial' PRF}
We define $\prf_k(x) = H(k || x)$ where $H:\bin^{*}\rightarrow\bin^{n(\secpar)}$ is a random oracle. Clearly this is a PRF; as the output of a random oracle, it is indistinguishable from a random function. Let $\sender$ be an unbounded oracle TM and $\receiver$ be an oracle PPTM where both have access to the random oracle $H$. \\

We will prove the following theorem.

\begin{theorem}[Communication complexity of OPRF, Perfect Completeness and Perfect Privacy]
	Let $\prf$ be a pseudorandom function as defined above, and $\sender$ and $\receiver$ have inputs as defined in $\F_\oprf$ respectively. Then any protocol $\Pi_\oprf$ which realizes $\F_\oprf$ with perfect correctness and perfect privacy in the $\F_\ot$-hybrid model must have total communication complexity proportional to $2^{m(\secpar)}$.
\end{theorem}

\paragraph{Brief Sketch.} Our argument proceeds as follows. Note that in order to evaluate the PRF at any point $x$, the oracle call $H(k||x)$ must be made. Clearly this oracle call cannot be made by the PPT receiver, since otherwise the receiver's view will consist of a polynomial-sized list of oracle queries to $H$ which contains $k||x$ -- this violates sender privacy as receiver learns $k$. Thus, this oracle call must be made by the sender.

Thus, the sender must make the oracle call $H(k||x)$. Note that by perfect correctness, the receiver must obtain this value regardless of the private randomness of the sender. Furthermore, this call 


%	\bibliographystyle{alpha}
%	\bibliography{references}
	
\end{document}









